% =========================================================================
% -------------------------------------------------------------------------
% Design:
% -------------------------------------------
%
%  This is a good place to outline key objectives of this section.
%
% -------------------------------------------------------------------------

\section{Design of Amanzi}

\subsection{High-Level Objectives and Design}

\subsection{Low-Level?}

Open to suggestion on how to separate and describe low-level design
concepts and implementations.

\subsection{Operators}
An Operator represents a map from linear space $X$ to linear space $Y$.
Typically, this map is a linear map; however, there are exceptions in
high-order advection methods. The spaces $X$ and $Y$ are described 
by class {\tt CompositeVector}. 
A few maps $X \to Y$ are supported now.
At the moment $X = Y$.

An Operator consists of, potentially, both a single, global, assembled
matrix AND an un-ordered, additive collection of lower-rank (or equal) 
local operators, here called 'Op's. During its construction, an operator
can grow by assimilating more Ops. An Operator knows how to peform 
the method "Apply" and assemble all of its local Ops.

Typically the forward operator is applied using only local Ops.
The inverse operator typically requires assembling a matrix, which 
may represent the entire operator or may be a Schur complement.

In all Operators, Ops, and matrices, a key concept is the schema. 
A schema includes, at the least, an enum representing the dofs associated
with the Operator's domain (and implied equivalent range, since $X=Y$).
A schema may also, in the case of Ops, include information on the base
entity on which the local matrix lives.

The idea behind the desing of Amanzi operators is to separate three 
functionalities that are frequently placed in a single class in other
C++ packages.

\begin{enumerate}
\item Assembly of a global matrix, including matrix map issues;
action of an inverse operator; calculation of Schur complements.

\item Data layout of local matrices, such as the local schema.

\item Values in local matrices, i.e. the actual discretization.
\end{enumerate}

The class {\tt Operator} performes actions summarized in item \#1. There exists
a series of inheriting classes such as {\tt Operator\_Cell}, {\tt Operator\_FaceCell}, 
{\tt Operator\_FaceCellSff}, where the suffix {\tt \_X} indicates the map/matrix.
They are derived from the virtual class {\tt Operator} which stores a list of Op classes,
see below.

A series of {\tt Op\_BASE\_DOFS} classes (such as {\tt Op\_Cell\_FaceCell})
handle data layout (item \#2). These are really just structs of vectors of
dense matrices of doubles, and simply provide a type.
They are derived from the virtual class {\tt Op}.

The classes {\tt OperatorDiffusion}, {\tt OperatorAdvection}, and {\tt OperatorAccumulation} 
in item \#3 create operators of the right type (for instance {\tt Operator\_FaceCell}), populate their 
values, and stick them in a {\it global operator}.
These are physics based operators and perform complex operations such as approximation
of Newton correction terms.

This desing enables a few things for future code development.
For instance, it should make creating surface matrices, and then assembling into a 
subsurface matrix doable by introducing a new {\tt Op} class.  
It also makes it trivial to assemble the global operator into a bigger, containing 
matrix (i.e. energy + flow) as any of the four sub-blocks.

The only potentially confusing part is the use of the visitor pattern (i.e. double dispatch 
in this case) to resolve all types.  
For instance to assemble a matrix, we may use the following pseudocode

\begin{verbatim}
// Operator
AssembleMatrix(Matrix A) {
  for each op {
    op->AssembleMatrix(this, Matrix A);
  }
}

virtual AssembleMatrixOp(Op_Cell_FaceCell& op) { 
  // throw error, not implemented
}

// Op
AssembleMatrix(Operator* global_op, Matrix& A) = 0;

// Op_Cell_FaceCell
AssembleMatrix(Operator* global_op, Matrix& A) {
  global_op->AssembleMatrixOp(*this, A);
}

// Operator_FaceCell
AssembleMatrixOp(Op_Cell_FaceCell& op, Matrix& A) {
  // This method now know both local schema and the matrix's dofs, 
  // and assembles the face+cell local matrices into the matrix.
}
\end{verbatim}

The reason for the double dispatch is to get the types specifically
without a ton of statement like this one "if (schema | XX \&\& schema | YY) 
\{ assemble one way \} else \{ assemble another way\}".

Note on the construction of Operators: For simple operations
(i.e. diffusion), Operators are not constructed directly.  Instead, 
a helper class that contains methods for creating and populating the
Ops within the Operator is created. This helper class can create the
appropriate Operator itself. More complex operations, i.e. 
advection-diffusion, can be generated by creating an Operator that 
is the union of all dof requirements, and then passing this Operator
into the helper's constructor. When this is done, the helper simply 
checks to make sure the Operator contains the necessary dofs and
adds local Ops to the Operator's list of Ops.

Note on implementation for developers: Ops work via a visitor pattern.
Matrix assembly, "Apply", apllication of boundary conditions, and symbolic assembly 
are implemented by the virtual class {\tt Operator} calling a dispatch to the 
virtual class {\tt Op}, which then dispatches back to the derived class Operator so that
type information of both the Operator (i.e. global matrix info) and 
the Op (i.e. local matrix info) are known.

Note on implementation for developers: Ops can be shared by
Operators. In combination with {\it CopyShadowToMaster()} and {\it Rescale()},
the developer has a room for a variaty of optimized implementations.
The key variable is ops\_properties. The key parameters are 
OPERATOR\_PROPERTY and described in Operators\_Defs.hh.





