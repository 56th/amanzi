% =========================================================================
% -------------------------------------------------------------------------
% Design:
% -------------------------------------------
%
%  This is a good place to outline key objectives of this section.
%
% -------------------------------------------------------------------------

\section{Design of Amanzi}

\subsection{High-Level Objectives and Design}

\subsection{Low-Level?}

Open to suggestion on how to separate and describe low-level design
concepts and implementations.

\subsection{Discrete PDEs and Operators}
An Operator represents a map from linear space $X$ to linear space $Y$.
Typically, this map is a linear map; however, it can be used also to calculate
a nonlinear residual. 
The spaces $X$ and $Y$ are described by class {\tt CompositeVector}. 
A few maps $X \to Y$ are supported now.

An discrete PDE consists of (a) a single global operatorr, (2) an 
optional assembled matrix, and (3) an un-ordered additive collection of 
lower-rank (or equal) local operators, here called Ops. 
During its construction, a PDE can grow by assimilating more Ops. 
The global Operator knows how to peform the method "Apply" and assemble 
all of its local Ops.
The PDE knows how to apply boundary conditions and to create a preconditioner.

Typically the forward operator is applied using only local Ops.
The inverse operator typically requires assembling a matrix, which 
may represent the entire operator or may be only its Schur complement.

In all Operators and Ops, a key concept is the schema. 
A schema includes at least one enum representing the dofs associated
with the Operator's domain and range. A single schema is a major limitation 
since it implies $X=Y$.
The new design (backward compatible) includes two schemas that are more 
detailed. A list of enums allows us to represent various collections of 
degrees of freedom including derivatives and vector components.
A schema includes also information on the base entity on which the local 
matrix lives.

The idea behind the desing of Amanzi operators is to separate three 
functionalities that are frequently placed in a single class in other
C++ packages.

\begin{enumerate}
\item Local matrices and data layout such as the local schema.

\item Operators: assembly of a global matrix and elemental operations with it
such as action of an inverse operator, and calculation of the Schur complement.

\item Discrete PDEs: populate values in local matrices, add nonlinear 
coefficients, create specialized preconditioners, and impose special
boundary conditions. 
\end{enumerate}

A series of {\tt Op\_BASE\_DOFS} classes (such as {\tt Op\_Cell\_FaceCell} and 
{\tt Op\_Cell\_Schema}) handle data layout (item \#1). 
These are really just structs of vectors of
dense matrices of doubles, and simply provide a type.
They are derived from the virtual class {\tt Op}.

The class {\tt Operator} performes actions summarized in item \#2. There exists
a series of inheriting classes such as {\tt Operator\_Cell}, {\tt Operator\_Schema}, 
{\tt Operator\_FaceCellSff}, where the suffix {\tt \_X} indicates the map/matrix.
Explicit maps are superceded by the flexible new schema.
They are derived from the virtual class {\tt Operator} which stores a local 
and global schemas. 
One Operator is marked as the global operator. It stores a list of Op classes
with compatible (equal or smaller) schemas.

The classes {\tt Diffusion}, {\tt Advection}, {\tt Elasticity},  
and {\tt Accumulation} in item \#3 create operators of the right type (for instance 
{\tt Operator\_FaceCell} or {\tt Operator\_Schema}), populate their values, and stick 
them in a global operator.
These are physics based operators and perform complex operations such as approximation
of Newton correction terms.

This desing enables a few things for future code development.
For instance, it should make creating surface matrices, and then assembling into a 
subsurface matrix doable by introducing a new {\tt Op} class with a simple schema
or using class {\tt Op\_Cell\_Schema} with a complex schema. 
It also makes it trivial to assemble the global operator into a bigger, containing 
matrix (i.e. energy + flow) as any of the four sub-blocks.
Finally, the new schema support rectangular matrices useful for copling with 
Stokes-type systems.

The only potentially confusing part is the use of the visitor pattern (i.e. double 
dispatch in this case) to resolve all types.  
For instance to assemble a matrix, we may use the following pseudocode

\begin{verbatim}
// Operator
AssembleMatrix(Matrix A) {
  for each op {
    op->AssembleMatrix(this, Matrix A);
  }
}

virtual AssembleMatrixOp(Op_Cell_FaceCell& op) { 
  // throw error, not implemented
}

// Op
AssembleMatrix(Operator* global_op, Matrix& A) = 0;

// Op_Cell_FaceCell
AssembleMatrix(Operator* global_op, Matrix& A) {
  global_op->AssembleMatrixOp(*this, A);
}

// Operator_FaceCell
AssembleMatrixOp(Op_Cell_FaceCell& op, Matrix& A) {
  // This method now know both local schema and the matrix's dofs, 
  // and assembles the face+cell local matrices into the matrix.
}
\end{verbatim}

The reason for the double dispatch is to get the types specifically
without a ton of statements like this one "if (schema | XX \&\& schema | YY) 
\{ assemble one way \} else \{ assemble another way\}".

\underline{Note on the discretization of PDEs}: Discretization of a simple 
PDE (i.e. diffusion) is not constructed directly. 
Instead, a helper class that contains methods for creating and populating 
the Ops within the Operator is created. This helper class can create the
appropriate discretization itself. More complex PDES, for instance the
advection-diffusion, can be discretized by creating a global Operator that 
is the union of all dofs requirements, and then passing this Operator
into the helper's constructor. When this is done, the helper simply 
checks to make sure the Operator contains the necessary dofs and
adds local Ops to the global Operator's list of Ops.

\underline{Three notes on implementation for developers}: 
1. Ops work via a visitor pattern.
Matrix assembly, "Apply", apllication of boundary conditions, and symbolic assembly 
are implemented by the virtual class {\tt Operator} calling a dispatch to the 
virtual class {\tt Op}, which then dispatches back to the derived class Operator so that
type information of both the Operator (i.e. global matrix info) and 
the Op (i.e. local matrix info) are known.

2. Application of boundary conditions is driven by a discrete PDEs class. 
Hence, each PDE object contributing to the global PDE object must apply
boundary conditions itself.

3. Ops can be shared by Operators. 
In combination with {\it CopyShadowToMaster()} and {\it Rescale()},
the developer has a room for a variaty of optimized implementations.
The key variable is ops\_properties. The key parameters are 
OPERATOR\_PROPERTY and described in Operators\_Defs.hh.





