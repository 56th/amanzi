\documentclass[12pt]{article}
\usepackage{color}
\usepackage{multirow}


% begin preamble
\setlength{\baselineskip}{16.0pt}    % 16 pt usual spacing between lines

\setlength{\parskip}{3pt plus 2pt}
\setlength{\parindent}{20pt}
\setlength{\oddsidemargin}{0.5cm}
\setlength{\evensidemargin}{0.5cm}
\setlength{\marginparsep}{0.75cm}
\setlength{\marginparwidth}{2.5cm}
\setlength{\marginparpush}{1.0cm}
\setlength{\textwidth}{150mm}

\pagestyle{empty} % use if page numbers not wanted

% end preamble

\begin{document}
\title{Amanzi Extern Software Requirements}

\maketitle


\section*{System Software Environment}
The libraries and tools listed in this section are assumed to be part of the target platform. Typically, 
a user would not build these libraries from source.  The Amanzi team recommends that users use
pre-built binaries or tools such as MacPorts, yum, rpm to install these libraries.  

At a minimum the build environment must have,
\begin{itemize}
\item A UNIX-like OS. Amanzi has been built on several flavors of Linux (Red Hat/CentOS, Ubuntu, Fedora) and 
Mac OS 10.5 and 10.6. Native Windows builds are not supported at this time.
\item C and C++ compiler.
\item MPI. Amanzi has been built against OpenMPI and MPICH. Amanzi does not support non-MPI builds at this time.
\item BLAS/LAPACK built and tuned to the target architecture.
\end{itemize}

If these libraries are installed in non-standard locations, user's will need to know the location 
of the header and library files to build most of the other third party libraries (TPLs).

\section*{Required External Software}
Amanzi leverages existing software libraries and  

\section*{Supported External Software}
\end{document}

