% =========================================================================
% -------------------------------------------------------------------------
% Thermal:
% -------------------------------------------
%
%  Non-isothermal (heat)
%
% -------------------------------------------------------------------------

\section{Thermal Processes}
\label{sec:thermal-processes}

\subsection{Overview}

Heat flow and thermal conduction is an important aspect of many geochemical systems 
affecting chemical processes through changes in equilibrium and kinetic rate constants. 
Non-isothermal conditions can also result in buoyancy driven flow leading to convection cells 
and causing fingering phenomena due to differences in density.

%---------------------------------------------------------------------------------------------
% Conservation Equations - Energy.
%---------------------------------------------------------------------------------------------

\subsection{Energy Conservation Equation}
\label{sec:Energy}


One form of the governing equation for energy conservation in a porous medium with porosity $\phi$ is given by
\BA\label{eq:energy_balance}
  \frac{\partial}{\partial t} \Big[ 
  \phi \rho_l u_l s_l
  + 
  (1-\phi )\rho_r u_r 
  \Big] \ 
  + \ &
  \boldsymbol{\nabla} \cdot \left[ 
  -\frac{\K k_{rl} }{\mu _l } (\boldsymbol{\nabla} p_l 
  -
  \rho _{\alpha} g \boldsymbol{e}_z)\rho _l h_l 
  - 
  Q_T 
  \right] 
  \eq 
  Q_e,
\EA
where
$\rho_{r}$ and $\rho_{r}$ are the density, 
$u_{r}$ and $u_l$ are the internal energy 
of the rock and the liquid phases, respectively;
$\K$ is the absolute permeability tensor,
$k_{rl}$ is the relative permeability coefficient,
$s_l$ is the liquid saturation,
$\mu_l$ is fluid viscosity,
$h_l$ is the enthalpy of the liquid phase, 
$Q_T$ represents thermal conduction and radiation, and 
$Q_{e} $ is a source or sink of energy. 
Note that the energy consumed/produced by chemical reactions is included in this last term. 


Assuming that the internal energy $u_r$ and the thermal conduction $Q_T$
have linear dependence on temperature,  
$$
  u_r = c_r T \qquad \text{and}\qquad Q_T=\kappa T,
$$
the equation \eqref{eq:energy_balance} takes the form 
\EQ
  \frac{\partial}{\partial t} 
  \Big[ \phi s_l \rho_l u_l 
  + 
  (1-\phi) \rho_r c_r T \Big] 
  + 
  \boldsymbol{\nabla}\cdot \Big[
  \bq_l \rho_l h_l 
  - 
  \kappa\boldsymbol{\nabla} T\Big] 
  \eq 
  Q_e,
\EN
where 
$T$ refers to temperature, 
$\bq_l$ is the Darcy velocity 
\EQ
  \bq_l 
  =  
  -\frac{\K k_{rl} }{\mu _l } 
  \left(
  \boldsymbol{\nabla} p_l 
  -
  \rho _{\alpha} g \boldsymbol{e}_z
  \right), 
\EN
the coefficient $\kappa$ denotes the thermal conductivity of the medium and 
$c_r$ refers to the specific heat of the porous medium. 
The internal energy and enthalpy are related by the equation
\EQ
  u_l \eq h_l -\frac{p_l}{\rho_l}.
\EN
Thermal conductivity if often described by the phenomenological relation given by \citet{somerton1974high}
\EQ\label{cond} 
  \kappa \eq \kappa_{\rm dry} + \sqrt{s_l^{}} (\kappa_{\rm sat} - \kappa_{\rm dry}), 
\EN 
where $\kappa_{\rm dry}$ and $\kappa_{\rm sat}$ are dry and fully saturated rock thermal conductivities, 
and $s_l$ denotes the saturation state of the liquid. 

\paragraph{Evapotranspiration. } 
Evapotranspiration models should include all the processes that convert water from the aqueous phase into water in the gaseous phase, 
i.e., water vapor.  
This should also account for evaporation from soil and plant surfaces and plant transpiration and include options 
where these components vary wiith soil properties and structure of the plant canopy.

\paragraph{Equation of State.}

For multicomponent system, equation of state (EOS) data is required for water and all the NAPL components.  
Typically these are EOS for pure substances that are combined for mixtures.  
A basic EOS relates density to pressure and temperature.  
The form for the EOS is typically cubic such as the Soave-Redlich-Kwong (SRK) or the
Peng-Robinson (PR) models.  
Tabular models are also in common use.

Mixture thermodynamics is used to combine pure phase EOS's for the application.  
The EPA lists over 80 potential NAPL components
(volatile organic compounds or VOCs) that can cause groundwater
contamination and has data bases with at least some properties for these
contaminants.


\paragraph{Mixture Internal Energy and Enthalpy.} 

These will generally follow the simple additive rule based of
component values and mass fraction. Consideration must be also given
to heats of solution.


\subsection{Data Needs}

Equations of state for fluid density, internal energy and/or enthalpy are needed 
in addition to heat capacity and thermal conductivity of the porous medium. 
Often the fluid properties for a complex mixture are unknown and the pure phase end member properties are used. 

\subsection{Boundary Conditions, Sources and Sinks}

Boundary conditions may take the form of specified temperature or heat flux including zero temperature gradient. 
Initial conditions include specifying the temperature over the computational domain 
such as a constant value or derived from the geothermal gradient.

\subsection{Coupling Considerations}

As heat flow is coupled to the Darcy flux, the heat equation itself is coupled to the flow equation 
as well as reactive transport equations through heat generated by chemical reactions. 
Conversely the flow and reactive transport equations are coupled to the heat equation 
through the temperature dependence of fluid properties such as density, viscosity, internal energy and enthalpy, 
and equilibrium thermodynamic and kinetic rate constants.

