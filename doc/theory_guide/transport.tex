%!TEX root = theory.tex
%
% =========================================================================
% -------------------------------------------------------------------------
% Transport (Advection - Dispersion):
% -------------------------------------------
%
%  This is a good place to outline key objectives of this section.
%
% -------------------------------------------------------------------------

\section{Transport Processes}    
\label{sec:transport-processes}

\subsection{Overview}

Transport is one of the most important process that needs to be accurately 
captured in the Environmental Management modeling tool set.  
In what follows, we use "Transport" to refer to the set of physical processes 
that lead to movement of dissolved and solid contaminants in the subsurface, 
treating the chemical and microbiological reactions that can affect the transport 
rate through a retardation effect as a separate set of processes.  
The principal transport processes to be considered are \emph{advection}, 
\emph{mechanical dispersion}, and \emph{molecular diffusion}.  
The equation for mass conservation of species $C$ can be written as
$$
  \frac{\partial ( \phi \sum_\alpha [s_{\alpha} C_{\alpha}] )}{\partial t } +
  \bnabla \cdot \bJ_{\text{adv}} \eq \bnabla \cdot \bJ_{\text{disp}} + \bnabla \cdot \bJ_{\text{diff}} + Q,
$$
where $C_\alpha$ is concentration of phase $\alpha$, 
$\bJ_{\text{adv}}$ is advective flux, $\bJ_{\text{disp}}$ is the dispersive flux, 
$\bJ_{\text{diff}}$ is the diffusive flux (often grouped with the dispersive flux), and  
$Q$ is the summation of various source terms which may include reactions.

The principal assumptions associated with the transport process models derive 
from the continuum treatment of the porous medium.  
Pore scale processes, including the resolution of variations in transport rates 
within individual pores or pore networks \citep{li2008scale,kang2006lattice,lichtner-kang-2007}, 
are generally not resolved, although some capabilities for treating multi-scale 
effects are discussed in Section~\ref{sec:transport-dual-porosity}.  
In general, it is assumed that within any one Representative Elementary Volume (REV) 
corresponding to a grid cell all transport rates are the same.  
It will be possible, however, to define overlapping continua with distinct transport 
rates, as in the case where the fracture network and rock matrix are represented as separate continua.

Transport processes may be tightly coupled to both flow and reaction processes.  
In the case of flow, one important coupling is associated with the transport of chemical 
constituents that affect the density of the solution, which in turn affects flow rates through buoyancy.
In the case of chemical reactions, the coupling effect is normally very strong for reactive constituents.  
Chemical reactions may consume components present in the gaseous phase (e.g., CO$_2$ or O$_2$), thus 
modifying the saturation of the phase itself.  
Or reactions can strongly modify gradients, and thus transport rates, by either consuming 
or producing various chemical species.

\begin{center}
\begin{longtable}{cp{7cm}c}
\caption{List of local variables.} \label{table:flow-list-of-variables} \\

\multicolumn{1}{c}{Symbol} & \multicolumn{1}{c}{Meaning} & \multicolumn{1}{c}{Units} \\
\hline  \hline 
\endfirsthead

\multicolumn{3}{c}{{\tablename} \thetable{} -- Continued} \\
\multicolumn{1}{c}{Symbol} & \multicolumn{1}{c}{Meaning} & \multicolumn{1}{c}{Units} \\
\hline  \hline 
\endhead

\hline \multicolumn{3}{c}{{Continued on next page}} \\ 
\hline \hline 
\endfoot

\hline \hline
\endlastfoot

$p_l$      & liquid pressure      &  $\upa$ \\
$\bq$      & Darcy velocity       &  $\um\ucdot\us^{-1}$  \\
$s_l$      & liquid saturation    &  $-$ \\
$\mu_l$    & liquid viscosity     &  $\upa\ucdot\us$ \\
$\rho_l$   & liquid density       &  $\ukg\ucdot\um^{-3}$ \\
$\phi$     & porosity             &  $-$  \\
$\phi_f$   & porosity of fracture &  $-$  \\
$\phi_m$   & porosity of matrix   &  $-$  \\
$\eta_l$   & molar liquid density &  $\umol\ucdot\um^{-3}$ \\

\end{longtable}
\end{center}


%%%%%%%%%%%%%%%%%%%%%%%%%%%%%%%%%%%%%%%%%%%%%%%%%%%%%%%%%%%%%%%%%%%%%%%%%%%%%%%%%%%%
%%%%%%%%%%%%%%%%%%%%%%%%%%%%%%%%%%%%%%%%%%%%%%%%%%%%%%%%%%%%%%%%%%%%%%%%%%%%%%%%%%%%
\subsection{Single-Phase Transport} 
\label{sec:transport-single-phase}


%%%%%%%%%%%%%%%%%%%%%%%%%%%%%%%%%%%%%%%%%%%%%%%%%%%%%%%%%%%%%%%%%%%%%%%%%%%%%%%%%%%%
\subsubsection{Process Model Equations} 

The transport in Amanzi neglects concentration of species in gaseous phase.
The simplified equation is
\begin{equation}  \label{eq:MassConservation}
  \frac{\partial ( \phi s_l C )}{\partial t } +
  \bnabla \cdot \bJ_{\text{adv}} \eq \bnabla \cdot \bJ_{\text{disp}} + \bnabla \cdot \bJ_{\text{diff}} + Q,
\end{equation}


%%%%%%%%%%%%%%%%%%%%%%%%%%%%%%%%%%%%%%%%%%%%%%%%%%%%%%%%%%%%%%%%%%%%%%%%%%%%%%%%%%%%
\subsubsection{Assumptions and Applicability}

We consider pure advection process and exclude the attenuation mechanisms and microbial 
behaviors that are discussed elsewhere. 
Note that there are situations where the advection could be modified by attributes of 
the transported mass and the pore structure.  
One is the potential for nonreactive anions to be repulsed by negatively charged 
solid surfaces into the center of pore throats where the velocity is faster.  
Another is the advection of inorganic and organic colloids, and microorganisms, 
whose movement can be affected by the geometry of pore throats.  
In addition to being subject to the same physicochemical phenomena as abiotic colloids, 
microorganisms have biological processes that can affect advection (e.g., temporal 
changes in surface properties due to changes in metabolic state; chemotaxis; predation).  


%%%%%%%%%%%%%%%%%%%%%%%%%%%%%%%%%%%%%%%%%%%%%%%%%%%%%%%%%%%%%%%%%%%%%%%%%%%%%%%%%%%%
\subsubsection{Advective Transport}  
\label{sec:transport-advection}

Advection is the process where the bulk fluid motion transports mass and heat.  
In the simplest conceptualization of advection, the mass of a component in a fluid 
parcel simply moves with the velocity of the fluid parcel.  
This assumes there are no other processes (e.g., diffusion, dispersion, reactions) 
that can affect the component concentration in the fluid parcel.  
Thus, advection can be a particulate or a dissolved species moving with the pore-water 
whose velocity is governed by the flow processes (discussed elsewhere).  
Continuum models have addressed these behaviors using bulk parameterizations 
to characterize the pore-scale controls and controlling chemical gradients.

Numerical difficulties with the accuracy, robustness, and computation efficiency 
of modeling the advection of moving steep concentration fronts, especially in 
complex velocity fields, are well known.  
In some cases, there are constraints on the Peclet and Courant numbers for the 
useful application of a given technique.
The advective flux, $\bJ_{\text{adv}}$, of a dissolved species is described 
mathematically as
\begin{equation} \label{eq:AdvectiveFlux} 
  \bJ_{\text{adv}} = \phi s_l \bv_l C_{i},  
\end{equation}
where $\phi$ is the porosity, $s_l$ is liquid saturation, $\bv_l$ is the 
average linear velocity of the liquid, and $C_i$ is the concentration of the $i$th species.


%%%%%%%%%%%%%%%%%%%%%%%%%%%%%%%%%%%%%%%%%%%%%%%%%%%%%%%%%%%%%%%%%%%%%%%%%%%%%%%%%%%%
\subsubsection{Dispersive Transport}   
\label{sec:transport-dispersion}

\paragraph{Isotropic Media.}
\paragraph{Anisotropic Media.}


%%%%%%%%%%%%%%%%%%%%%%%%%%%%%%%%%%%%%%%%%%%%%%%%%%%%%%%%%%%%%%%%%%%%%%%%%%%%%%%%%%%%
\subsubsection{Diffusive Transport} 
\label{sec:diffusive-transport}

\paragraph{General Formulation for Molecular Diffusion.}

\paragraph{Single Species Diffusion (Fick's Law).}


%%%%%%%%%%%%%%%%%%%%%%%%%%%%%%%%%%%%%%%%%%%%%%%%%%%%%%%%%%%%%%%%%%%%%%%%%%%%%%%%%%%%
%%%%%%%%%%%%%%%%%%%%%%%%%%%%%%%%%%%%%%%%%%%%%%%%%%%%%%%%%%%%%%%%%%%%%%%%%%%%%%%%%%%%
\subsection{Boundary Conditions, Sources and Sinks} 
\label{sec:transport-boundary-conditions}


%%%%%%%%%%%%%%%%%%%%%%%%%%%%%%%%%%%%%%%%%%%%%%%%%%%%%%%%%%%%%%%%%%%%%%%%%%%%%%%%%%%%
%%%%%%%%%%%%%%%%%%%%%%%%%%%%%%%%%%%%%%%%%%%%%%%%%%%%%%%%%%%%%%%%%%%%%%%%%%%%%%%%%%%%
\subsection{Single-Phase Transport with Dual Porosity Model} 
\label{sec:transport-single-phase-dual-porosity}

The multiscale nature of porous media and the transport processes associated is 
arguably the most significant and largely unresolved challenge for simulation of 
fate and transport in subsurface aquifers.  
Transport actually operates at the pore scale where variations in flow velocity 
and reaction rates can result in microscopic variability in transport rates.  
Continuum treatments of transport in porous media cannot resolve such sub-grid 
variations easily, although various upscaling techniques may be available for 
capturing the smaller scale behavior.  
In addition, multi-continuum or hybrid approaches may obviate the need for a formal 
upscaling procedure, although there are significant computational difficulties and 
expense associated with their implementation.


%%%%%%%%%%%%%%%%%%%%%%%%%%%%%%%%%%%%%%%%%%%%%%%%%%%%%%%%%%%%%%%%%%%%%%%%%%%%%%%%%%%%
\subsubsection{Process Model Equations} 

The dual porosity formulation of the solute transport consists of two equations
for the fracture and matrix regions. 
In the fracture region, we have \citep{simunek-vangenuchten_2008}
\begin{equation}  \label{eq:MassConservation}
  \frac{\partial (\phi_f s_{lf} C_f)}{\partial t} 
  + \bnabla \cdot \bJ_{\text{adv}} \eq \bnabla \cdot \bJ_{\text{disp}} 
  + \bnabla \cdot \bJ_{\text{diff}} - \Sigma_s + Q_f,
\end{equation}
where $s_{lm}$ is liquid saturation in fracture, 
$\Sigma_s$ is the solute exchange term,
and $Q_f$ is source or sink term.
In the matrix region, we have
$$
  \frac{\partial (\phi_m s_{lm} C_m)}{\partial t}
  = \Sigma_s + Q_m,
$$
where $s_{lm}$ is liquid saturation in matrix, $Q_m$ is source or sink term.
The solute exchange term is defined as
$$
  \Sigma_s = \alpha_s (C_f - C_m) + \Sigma_w C^*,
$$
where $C^*$ is equal to $C_f$ if $\Sigma_w > 0$ and $C_m$ is $\Sigma_w < 0$.
The coefficient $\alpha_s$ is the first-order solute mass transfer coefficient [$\us^{-1}$].
